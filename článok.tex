\documentclass[12pt, a4paper]{article}
\usepackage{graphicx}
\setlength\parindent{24pt}
\begin{document}
\title{Využitie umelej inteligencie v stolových hrách}
\author{Andrej Streicher}
\date{November 2022}
\maketitle
\begin{abstract}
Cieľom tohto projektu je analýza histórie a vývoja umelej inteligencie v stolných hrách a jej dopad na všeobecný rozvoj umelej inteligencie. 
V tejto práci sa dozvieme o legendárnych počítačoch ako je DeepBlue, ktorý ako prvý porazil svetového šampióna, Garriho Kasparova a AlphaGo, ktorý ako prvý porazil ľudského hráča, Fan Hui-ho, a neskôr jedného z najlepších hráčov Go na svete Lee Sedol-a. Taktiež sa dozvieme o najnovších špičkových umelých inteligenciách, ktoré sa dokážu naučiť hrať hocijakú stolnú hru, bez manuálneho naprogramovania pravidiel hry.  
Taktiež sa dozvieme o rôznych technických metódach, ktorými sa umelá inteligencia učí, ako napríklad strojové učenie alebo umelé neurónové siete, alebo ktorými si vyberá medzi rozhodnutiami, ako napríklad Monte Carlo tree search (MCTS).
\end{abstract}
\section{Umelá inteligencia a jej vývin}
Umelá inteligencia ako pojem v počítačovej vede bol prvý krát predstavený na konferencií Darthmouth v roku 1956 Johnom McCarthym\cite{Corduck}. Na tejto istej konferencií debutoval prvý program, ktorý využíval automatizované uvažovanie, "Logic Theorist". Tento program bol schopný dokázať 38 z 52 matematických teorémov z knihy \textit{Principia Mathematica}. Jeden z dôkazov bol dokonca považovaný za elegantnejší než ručne vypočítaný originál z knihy. Táto konferencia sa taktiež považuje za "narodenie" umelej intelgencie ako vedy.\\
\section{Ako sa umelá inteligencia učí}
\section{Umelá inteligencia a stolové hry}
Stolové hry boli súčasťou vývinu umelej inteligencie 
\section{Deep Blue}
\section{AlphaGo}
\section{Moderné umelé inteligencie v stolových hrách}
\bibliography{myBib}{}
\bibliographystyle{plain}
\end{document}
