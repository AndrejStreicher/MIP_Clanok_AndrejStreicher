\chapter{Umelá inteligencia a jej vývin}
Umelá inteligencia ako pojem v počítačovej vede bol prvý krát predstavený na konferencií Darthmouth v roku 1956 Johnom McCarthym. Na tejto istej konferencií debutoval prvý program, ktorý využíval automatizované uvažovanie, "Logic Theorist". Tento program bol schopný dokázať 38 z 52 matematických teorémou z knihy \textit{Principia Mathematica}. Jeden z dôkazov bol dokonca považovaný za elegantnejší než ručne vypočítaný originál z knihy\cite{Corduck}. Táto konferencia sa taktiež považuje za 'narodenie' umelej inteligencie ako vedy. Logic Theorist predstavil niekoľko konceptov ktoré sa stali kľúčovými pre výskum umelej inteligencie:
\section{Vyhľadávací strom}
\begin{wrapfigure}{r}{0.20\textwidth} %this figure will be at the right
    \centering
    \vspace{-30pt}
    \includegraphics[width=0.35\textwidth]{Obrázky/BinarySearchTree.png}
    \caption{Binárny strom}
\end{wrapfigure} 
Logic Theorist prechádzal cez binárny vyhľadávací strom.
Koreňom stromu bola počiatočná hypotéza a každá vetva bola dedukcia založená na pravidlách logiky. Cieľom programu bolo dostať sa k výroku, ktorý sa snažil dokázať. Všetky kroky, cez ktoré program prešiel tvoria dôkaz – sériu tvrdení, ktoré viedli od hypotézy k výroku, ktorý mal dokázať\cite{Corduck}.
\section{Heuristika}  
Newell a Simon\footnote{Dvaja z tvorcov Logic Theorist-a} si uvedomili, že ich vyhľadávací strom bude exponenciálne rásť a že budú potrebovať ‘orezať‘ niektoré vetvy, ktorých cesty pravdepodobne nepovedú k riešeniu. 

Pravidlá tohto orezávania nazvali „heuristika“, termín ktorý bol už v matematike používaný. Heuristika sa stala dôležitou oblasťou výskumu umelej inteligencie a zostáva dôležitou metódou na prekonanie exponenciálneho rastu vyhľadávacích stromov.
\section{Spracovanie zoznamov}
Na implementáciu Logic Theorist-u na počítač vyvinuli jeho tvorcovia programovací jazyk IPL\footnote{Information Processing Language}, ktorý používal rovnakú formu spracovania symbolických zoznamov, ktorá neskôr tvorila základ McCarthyho programovacieho jazyka Lisp, dôležitého jazyka, ktorý stále používajú výskumníci umelej inteligencie\cite{Corduck}.
\section*{}
\vspace{-20pt}
Pamela McCorduck\footnote{Spisovateľka a žurnalistka so zameraním na umelú inteligenciu} povedala o Logic Theorist-ovi: 

\textit{„... bol dôkazom, že stroj môže vykonávať úlohy, ktoré sa doteraz považovali za inteligentné, kreatívne a jedinečne ľudské“}\cite{Corduck}.


Logic Theorist predstavuje míľnik vo vývoji umelej inteligencie a nášho chápania inteligencie vo všeobecnosti.